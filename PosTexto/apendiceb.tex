%%%% APÊNDICE B
%%
%% Texto ou documento elaborado pelo autor, a fim de complementar sua argumentação, sem prejuízo da unidade nuclear do trabalho.

%% Título e rótulo de apêndice (rótulos não devem conter caracteres especiais, acentuados ou cedilha)
\chapter{Código Fonte do Fluxo de Integração Contínua}\label{cap:apendiceb}

O código da \autoref{codigo:ciyaml} apresenta as etapas que compõem o fluxo de integração contínua do projeto. O arquivo é escrito em \textit{YAML} e é responsável por automatizar a execução de tarefas como instalação de dependências, execução de testes, análise estática de código e envio de relatórios de cobertura de testes para o serviço de terceiros \textit{Codecov}, que é um serviço de acompanhamento de cobertura de testes.

\begin{sourcecode}[htb]
\caption{\label{codigo:ciyaml}Arquivo de configuração do fluxo de integração contínua}
\begin{lstlisting}[frame=single]
name: CI

on: ['push', 'pull_request']

concurrency:
    group: '${{ github.workflow }}-
        ${{ github.event.pull_request.number || github.ref }}'
    cancel-in-progress: true

jobs:
    continuous-integration:
    runs-on: ubuntu-latest
    steps:
        - name: Checkout
        uses: actions/checkout@v4
        with:
            fetch-depth: 0

        - name: Setup Node.js
        uses: actions/setup-node@v4
        with:
            node-version-file: '.nvmrc'
            cache: 'npm'
            cache-dependency-path: |
            package-lock.json
            packages/*/package-lock.json

        - name: Install dependencies, typecheck, lint, format, test
        run: |
            npm ci
            npm run typecheck
            npm run lint
            npx prettier --check .
            npx jest --ci --coverage
        env:
            CI: true

        - name: Upload coverage reports to Codecov
        uses: codecov/codecov-action@v4.0.1
        with:
            token: ${{ secrets.CODECOV_TOKEN }}
    
\end{lstlisting}
\fonte{}
\end{sourcecode}