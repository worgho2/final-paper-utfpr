%%%% GLOSSÁRIO
%%
%% Relação de palavras ou expressões técnicas de uso restrito ou de sentido obscuro, utilizadas no texto, acompanhadas das
%% respectivas definições.

%% Entradas do glossário: \newglossaryentry{rótulo}{informações da entrada}

\newglossaryentry{ec2SpotInstance}{
  name = {\gls{ec2} \textit{Spot Instance}},
  plural = {\gls{ec2} \textit{Spot Instances}},
  sort = {EC2 Spot Instance},
  description = {é um tipo de instância de máquina virtual oferecida pela \gls{aws} que permite ao usuário solicitar capacidade computacional a preços reduzidos em comparação com as instâncias sob demanda}
}

\newglossaryentry{guacamole}{
  name = {Apache Guacamole\texttrademark},
  sort = {Apache Guacamole},
  description = {é um cliente de desktop remoto baseado em navegador que suporta protocolos de desktop remoto como \gls{vnc}, \gls{rdp} e \gls{ssh}}
}

\newglossaryentry{desktop}{
  name = {\textit{desktop}},
  plural = {\textit{desktops}},
  sort = {desktop},
  description = {é um ambiente gráfico de trabalho que permite ao usuário interagir com o sistema operacional por meio de uma interface gráfica}
}

\newglossaryentry{onPremise}{
  name = {\textit{On-Premise}},
  sort = {On-Premise},
  description = {é um modelo de implantação de software em que o software é instalado e executado em servidores locais da organização do usuário, em vez de servidores remotos ou em nuvem pública}
}

\newglossaryentry{port}{
  name = {\textit{port}},
  plural = {\textit{ports}},
  sort = {port},
  description = {é um termo utilizado dentro da arquitetura hexagonal para se referir a uma interface utilizada na definição das regras de negócio dentro de um caso de uso}  
}

\newglossaryentry{adapter}{
  name = {\textit{adapter}},
  plural = {\textit{adapters}},
  sort = {adapter},
  description = {é um termo utilizado dentro da arquitetura hexagonal para se referir a uma classe que implementa uma interface utilizada na definição das regras de negócio dentro de um caso de uso}
}




% \newglossaryentry{pai}{%% Informações da entrada
%   name        = {pai},
%   plural      = {pais},
%   description = {um exemplo de entrada pai que possui subentradas (entradas filhas)}
% }

% \newglossaryentry{componente}{%% Informações da entrada
%   name        = {componente},
%   plural      = {componentes},
%   parent      = {pai},
%   description = {um exemplo de uma entrada componente, subentrada da entrada chamada \gls{pai}}
% }

% \newglossaryentry{filho}{%% Informações da entrada
%   name        = {filho},
%   plural      = {filhos},
%   parent      = {pai},
%   description = {um exemplo de uma entrada filha (subentrada) da entrada chamada \gls{pai}. Trata-se de uma entrada irmã da entrada chamada \gls{componente}}
% }

% \newglossaryentry{equilibrio}{%% Informações da entrada
%   name        = {equilíbrio da configuração},
%   see         = [veja também]{componente},
%   description = {uma consistência entre os \glspl{componente}}
% }

% \newglossaryentry{tex}{%% Informações da entrada
%   name        = {\TeX},
%   sort        = {TeX},
%   description = {é um sistema de tipografia criado por Donald E. Knuth}
% }

% \newglossaryentry{latex}{%% Informações da entrada
%   name        = {\latex},
%   sort        = {LaTeX},
%   description = {um conjunto de macros para o processador de textos \gls{tex}, utilizado amplamente para a produção de textos matemáticos e científicos devido à sua alta qualidade tipográfica}
% }

