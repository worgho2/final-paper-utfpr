%%%% RESUMO
%%
%% Apresentação concisa dos pontos relevantes de um texto, fornecendo uma visão rápida e clara do conteúdo e das conclusões do
%% trabalho.

\begin{resumoutfpr}

O acesso a recursos computacionais de alto desempenho é essencial para a realização de pesquisas acadêmicas e científicas, porém, a aquisição e manutenção de \glspl{desktop} de alto desempenho é um desafio para instituições públicas de ensino superior devido a existência de processos licitatórios complexos e demorados. Por outro lado, universitários advindos de famílias de baixa renda, que não possuem condições financeiras para adquirir um computador pessoal, podem ter dificuldades para realizar atividades acadêmicas dentro e fora da universidade. Sendo assim, o objetivo deste trabalho é desenvolver um sistema de \gls{vdi} capaz de possibilitar que alunos e pesquisadores de instituições de ensino superior acessem \glspl{desktop} virtuais alocados em nuvem pública através de qualquer dispositivo com acesso à internet e com suporte a um navegador web, aprimorando a gestão dos recursos computacionais mais potentes, uma vez que são consumidos como serviço ao invés de serem adquiridos através de processos licitatórios complexos e demorados.


%  %% Ambiente resumoutfpr

%  O resumo deve ressaltar de forma sucinta o conteúdo do trabalho, incluindo justificativa, objetivos,
%  metodologia, resultados e conclusão. Deve ser redigido em um único parágrafo, justificado, contendo
%  de 150 até 500 palavras. Evitar incluir citações, fórmulas, equações e símbolos no resumo.
 
%  A referência no resumo é elemento opcional em trabalhos acadêmicos, sendo que na UTFPR adotamos por
%  não incluí-la nos resumos contidos nos próprios trabalhos. As palavras-chave e as keywords são grafadas
%  em inicial minúscula quando não forem nome próprio ou nome científico e separados por ponto e vírgula.
 
%  % Add a blank line here
 

\end{resumoutfpr}

