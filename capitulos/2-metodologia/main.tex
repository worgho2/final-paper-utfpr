%%%% CAPÍTULO 3 - METODOLOGIA

\chapter{Metodologia}
\label{cap:metodologia}

Este capítulo aborda as metodologias de modelagem e desenvolvimento utilizadas no trabalho.

\section{Escopo do sistema}
\label{sec:escopoDoSistema}

O sistema \textit{Virtual Lab} deve possibilidar a criação e gerenciamento de instâncias de máquinas virtuais para uso em laboratórios remotos de ensino e pesquisa.
O acesso deve ser feito por meio de um navegador de internet, permitindo a utilização de qualquer dispositivo com um navagador de internet com acesso à rede.

O sistema deve acomodar dois tipos de usuários: usuários comuns e administradores.
Os usuários comuns devem ter permissão de criar instâncias de máquinas virtuais com base em templates pré-definidos, gerenciar suas próprias instâncias, bem como o conteúdo instalado nas mesmas.
Os administradores devem ter as mesmas permissões dos usuários comuns, além de gerenciar os templates de instância, controlar o acesso dos usuários comuns ao sistema e aos recursos de hardware.

O sistema deve permitir o acesso aos usuários através de autenticação por meio de usuário e senha, bem como através de um provedor de identidade externo, caso o ambiente de uso do sistema já possua um provedor de identidade.

Inicialmente, um usuário administrador deve criar os templates de instância indicando o sistema operacional e a quantidade de armazenamento disponível.
Os usuários comuns devem então criar instâncias com base nesses templates, escolhendo entre os tipos de hardware permitidos.

Ao ser criada, a instância deve passar por um processo de configuração inicial para garantir que o sistema operacional esteja pronto para uso, e então ser disponibilizada para o usuário, que poderá acessá-la atráves da página de conexões do sistema. As instâncias devem ser encerradas automaticamente após um período de inatividade.

\section{Modelagem}
\label{sec:modelagemDoSistema}

TODO

\section{Ferramentas e Tecnologias}
\label{sec:ferramentasETecnologias}

A \autoref{tab:ferramentasETecnologiasUtilizadas} apresenta uma lista de ferramentas, tecnologias e serviços utilizados na implementação do sistema. As versões utilizadas estão escritas de acordo com o padrão de versionamento semântico. (TODO: referenciar)

% Make de last column to be as wide as possible wrapping the text
\begin{longtable}{p{0.25\linewidth} p{0.15\linewidth} p{0.525\linewidth}}%% Ambiente longtable
\caption{Ferramentas e tecnologias utilizadas\label{tab:ferramentasETecnologiasUtilizadas}} \\%% Legenda e rótulo
\toprule
\textbf{Nome} & \textbf{Versão} & \textbf{Descrição} \\
\midrule
\endfirsthead%% Encerra cabeçalho da primeira página
\caption[]{Ferramentas e tecnologias utilizadas} \\%% Legenda
\multicolumn{3}{r}{\textbf{(continuação)}} \\
\toprule
\textbf{Nome} & \textbf{Versão} & \textbf{Descrição} \\
% \midrule
\endhead%% Encerra cabeçalho das demais páginas
% \midrule
\multicolumn{3}{r}{\textbf{(continua)}} \\
\endfoot%% Encerra rodapé das demais páginas
% \bottomrule
\\[-0.5\linha]
\caption*{\nomefonte: Autoria própria (2024)} \\
\endlastfoot%% Encerra rodapé da última página
Node.js & \textsuperscript{$\wedge$}18.19.0 & Ambiente de execução de javascript utilizado em todos os serviços de backend \\

\hline

ECMAScript & ES2020 & Especificação formal da linguagem javascript a qual o código escrito é compatível \\

\hline

Typescript & {$\leq$}5.4.0 & Extensão do Javascript que adiciona suporte para tipagem, utilizada na implementação de todos os serviços de backend e do cliente web do sistema \\

\hline

Npm & 10.2.3 & Gerenciador de dependências do Node.js \\

\hline

Shell Script & \gls{n/a} & Linguagem de scripts utilizada para implementar a configuração automatizada das instâncias baseadas em LINUX \\

\hline

Windows PowerShell Script & 5 & Linguagem de scripts utilizada para implementar a configuração automatizada das instâncias baseadas em WINDOWS \\

\hline

Apache Velocity Template Language & \gls{n/a} & Engine de templates utilizada para definir as permissões de conexão dos usuários ao serviços de notificações do servidor ao cliente \\

\hline

Markdown & \gls{n/a} & Linguagem de marcação de texto utilizada na documentação do sistema \\

\hline

OpenApi & 3.0.0 & Especificação para documentação de API. \\

\hline

Zod & \textsuperscript{$\wedge$}3.23.8 & Biblioteca utilizada para validar o conteúdo das requisições dos serviços \\

\hline

Prettier & \textsuperscript{$\wedge$}3.3.1 & Utilitário utilizado para garantir a estilização do código. \\

\hline

Jest & \textsuperscript{$\wedge$}29.7.0 & Framework de teste para javascript \\

\hline

Eslint & \textsuperscript{$\wedge$}8.56.0 & Utilitário de análise estática de código. \\

\hline

AWS CDK & 2.142.1 & Framework utilizado para definição dos componentes de infraestrutura como código \\

\hline

SST & 2.43.0 & Framework para construção de aplicações full-stack na AWS utilizando infraestrutura como código \\

\hline

Docusaurus & \textsuperscript{$\wedge$}3.4.0 & Gerador de documentação em formato de site estático. \\

\hline

React & \textsuperscript{$\wedge$}18.0.0 & Biblioteca utilizada no desenvolvimento do cliente web do sistema \\

\hline

Stoplight Elements & \textsuperscript{$\wedge$}8.3.1 & Biblioteca utilizada para renderização da documentação da API. \\

\hline

docker & 24.0.5 & Ferramenta utilizada para a criação do ambiente isolado do gateway de conexão \\

\hline

Chakra UI & \textsuperscript{$\wedge$}2.8.2 & Biblioteca de componentes de interface utilizada no cliente web. \\

\hline

Apache Guacamole & 1.5.3 & Utilizado para intermediar a conexão entre o cliente web e gateway de conexão. \\

\hline

Vite & \textsuperscript{$\wedge$}5.0.12 & Ferramenta responsável por gerenciar a implementação e o empacotamento do cliente web \\

\hline

React Icons & \textsuperscript{$\wedge$}5.0.1 & Biblioteca de Icones utilizada no cliente web \\

\hline

MongoDB Atlas & sdk \textsuperscript{$\wedge$}6.0.3 & Serviço de banco de dados não relacional utilizado para armazenar todos os dados do sistema. \\

\hline

AWS Amplify & sdk \textsuperscript{$\wedge$}6.0.16 & Biblioteca com integrações facilitadas, uitilizada no cliente web para autenticação e conexão com AWS AppSync \\

\hline

AWS Lambda PowerTools & \textsuperscript{$\wedge$}2.1.1 & Biblioteca utilizada para integrar a geração de logs dos serviços com o AWS CloudWatch \\

\hline

AWS AppSync & sdk \textsuperscript{$\wedge$}3.592.0 & Serviço utilizado para a comunicação em tempo real entre os serviçoes de backend e o cliente web \\

\hline

AWS Systems Manager Parameter Store & sdk \textsuperscript{$\wedge$}3.592.0 & Serviço utilizado para armazenar as configurações do sistema. \\

\hline

AWS Event Bridge & sdk \textsuperscript{$\wedge$}3.592.0 & Serviço de gerenciamento de eventos utilizado como barramento de comunicação entre os serviços de backend. \\

\hline

AWS CloudFormation & sdk \textsuperscript{$\wedge$}3.592.0 & Serviço utilizado para armazenar a definição dos componentes de infraestrutura como código \\

\hline

AWS Cognito User Pool & sdk \textsuperscript{$\wedge$}3.592.0 & Serviço de gerenciamento de usuários e credenciais. Ele foi utilizado para gerenciar dados sensíveis, mantendo somente dados operacionais na integração com o banco de dados. \\

\hline

AWS Event Bridge Scheduler & sdk \textsuperscript{$\wedge$}3.592.0 & Serviço de agendamento de operações utilizado no sistema de desligamento automático de instâncias osciosas \\

\hline

AWS Service Catalog & sdk \textsuperscript{$\wedge$}3.592.0 & Serviço de gerenciamento de modelos de IaC utilizado para o controle do ciclo de vida da instância e seus recursos dependentes \\

\hline

AWS CloudWatch & sdk \textsuperscript{$\wedge$}3.592.0 & Serviço de monitoramento utilizado para agregar os logs dos sistemas. \\

\hline

AWS Lambda & sdk \textsuperscript{$\wedge$}3.592.0 & Serviço de computação serverless utilizado para executar o serviço de API e tratamento de eventos do sistema \\

\hline

AWS Api Gateway & sdk \textsuperscript{$\wedge$}3.592.0 & Serviço gerenciado para criação de APIs serverless. \\

\hline

AWS Identity and Access Management & sdk \textsuperscript{$\wedge$}3.592.0 & Serivço utilizado para delimitação de todas as permissões atreladas aos componentes de infraestrutura do sistema \\

\hline

AWS S3 & sdk \textsuperscript{$\wedge$}3.592.0 & Serviço de armazenamento de objetos utilizado para servir o cliente web e armazenar todos os arquivos imutaveis do sistema \\

\hline

AWS CloudFront & sdk \textsuperscript{$\wedge$}3.592.0 & Serviço de distribuição de conteúdo utilizado como camada distribuída de cache para os serviços \\

\hline

AWS Certificate Manager & sdk \textsuperscript{$\wedge$}3.592.0 & Serviços utilizado para o gerenciamento dos certificados de domínio atrelados ao cliente web e à documentação \\

\hline

AWS Application Load Balancer & sdk \textsuperscript{$\wedge$}3.592.0 & Balanceador de carga gerenciado utilizado para gerenciar as conexões entre o cliente web e o cluster de containers rodando o Gateway de conexão. \\

\hline

AWS Elastic Container Service & sdk \textsuperscript{$\wedge$}3.592.0 & Serviço utilizado para a implantação do gateway de conexão, no formato de cluster de containeres com escalabilidade previsível \\

\hline

AWS Elastic Container Registry & sdk \textsuperscript{$\wedge$}3.592.0 & Serviço utilizado para armazenar as imagens de container do gateway de conexão \\

\hline

AWS Simple Notification Service & sdk \textsuperscript{$\wedge$}3.592.0 & Serviço de publisher Subscriber utilizado para a transacionar as mensagens emitidas na criação de instâncias \\

\hline

AWS EC2 & sdk \textsuperscript{$\wedge$}3.592.0 & Serviço central utilizado para gerenciar as instâncias, regras de conexão e as imagens de sistema operacional \\

\hline

\end{longtable}

\section{Implementa\c{c}\~ao}
\label{sec:implementacao}



TODO

\section{Testes e Valida\c{c}\~ao}
\label{sec:testesEValidacao}

TODO

\section{Implanta\c{c}\~ao do sistema}
\label{sec:implantacaoDoSistema}

