%%%% CAPÍTULO 3 - METODOLOGIA

\chapter{Metodologia}
\label{cap:metodologia}

Este capítulo aborda as metodologias de modelagem e desenvolvimento utilizadas no trabalho.

\section{Escopo do sistema}
\label{sec:escopoDoSistema}

O sistema \textit{Virtual Lab} deve possibilidar a criação e gerenciamento de instâncias de máquinas virtuais para uso em laboratórios remotos de ensino e pesquisa.
O acesso deve ser feito por meio de um navegador de internet, permitindo a utilização de qualquer dispositivo com um navagador de internet com acesso à rede.

O sistema deve acomodar dois tipos de usuários: usuários comuns e administradores.
Os usuários comuns devem ter permissão de criar instâncias de máquinas virtuais com base em templates pré-definidos, gerenciar suas próprias instâncias, bem como o conteúdo instalado nas mesmas.
Os administradores devem ter as mesmas permissões dos usuários comuns, além de gerenciar os templates de instância, controlar o acesso dos usuários comuns ao sistema e aos recursos de hardware.

O sistema deve permitir o acesso aos usuários através de autenticação por meio de usuário e senha, bem como através de um provedor de identidade externo, caso o ambiente de uso do sistema já possua um provedor de identidade.

Inicialmente, um usuário administrador deve criar os templates de instância indicando o sistema operacional e a quantidade de armazenamento disponível.
Os usuários comuns devem então criar instâncias com base nesses templates, escolhendo entre os tipos de hardware permitidos.

Ao ser criada, a instância deve passar por um processo de configuração inicial para garantir que o sistema operacional esteja pronto para uso, e então ser disponibilizada para o usuário, que poderá acessá-la atráves da página de conexões do sistema. As instâncias devem ser encerradas automaticamente após um período de inatividade.

\section{Modelagem}
\label{sec:modelagemDoSistema}

TODO

\section{Ferramentas e Tecnologias}
\label{sec:ferramentasETecnologias}

A \autoref{tab:ferramentasETecnologiasUtilizadas} apresenta uma lista de ferramentas, tecnologias e serviços utilizados na implementação do sistema. As versões utilizadas estão escritas de acordo com o padrão de versionamento semântico. \citep{semverdocs}

% Make de last column to be as wide as possible wrapping the text
\begin{longtable}{p{0.25\linewidth} p{0.15\linewidth} p{0.525\linewidth}}%% Ambiente longtable
\caption{Ferramentas e tecnologias utilizadas\label{tab:ferramentasETecnologiasUtilizadas}} \\%% Legenda e rótulo
\toprule
\textbf{Nome} & \textbf{Versão} & \textbf{Descrição} \\
\midrule
\endfirsthead%% Encerra cabeçalho da primeira página
\caption[]{Ferramentas e tecnologias utilizadas} \\%% Legenda
\multicolumn{3}{r}{\textbf{(continuação)}} \\
\toprule
\textbf{Nome} & \textbf{Versão} & \textbf{Descrição} \\
% \midrule
\endhead%% Encerra cabeçalho das demais páginas
% \midrule
\multicolumn{3}{r}{\textbf{(continua)}} \\
\endfoot%% Encerra rodapé das demais páginas
% \bottomrule
\\[-0.5\linha]
\caption*{\nomefonte: Autoria própria (2024)} \\
\endlastfoot%% Encerra rodapé da última página
Node.js \citep{nodejsdocs} & \textsuperscript{$\wedge$}18.19.0 & Ambiente de execução de javascript utilizado em todos os serviços de backend \\

\hline

ECMAScript \citep{ecmascriptdocs} & ES2020 & Especificação formal da linguagem javascript a qual o código escrito é compatível \\

\hline

Typescript \citep{typescriptdocs} & {$\leq$}5.4.0 & Extensão do Javascript que adiciona suporte para tipagem, utilizada na implementação de todos os serviços de backend e do cliente web do sistema \\

\hline

NPM \citep{npmdocs} & 10.2.3 & Gerenciador de dependências do Node.js \\

\hline

Shell Script \citep{shellscriptdocs} & \gls{n/a} & Linguagem de scripts utilizada para implementar a configuração automatizada das instâncias baseadas em LINUX \\

\hline

Windows PowerShell Script \citep{windowspowershelldocs} & 5 & Linguagem de scripts utilizada para implementar a configuração automatizada das instâncias baseadas em WINDOWS \\

\hline

Apache Velocity Template Language \citep{apachevelocitydocs} & 2.3 & Engine de templates utilizada para definir as permissões de conexão dos usuários ao serviços de notificações do servidor ao cliente \\

\hline

Markdown \citep{markdowndocs} & \gls{n/a} & Linguagem de marcação de texto utilizada na documentação do sistema \\

\hline

OpenAPI \citep{openapidocs} & 3.0.0 & Especificação para documentação de API. \\

\hline

Zod \citep{zoddocs} & \textsuperscript{$\wedge$}3.23.8 & Biblioteca utilizada para validar o conteúdo das requisições dos serviços \\

\hline

Prettier \citep{prettierdocs} & \textsuperscript{$\wedge$}3.3.1 & Utilitário utilizado para garantir a estilização do código. \\

\hline

Jest \citep{jestdocs} & \textsuperscript{$\wedge$}29.7.0 & Framework de teste para javascript \\

\hline

Eslint \citep{eslintdocs} & \textsuperscript{$\wedge$}8.56.0 & Utilitário de análise estática de código. \\

\hline

AWS CDK \citep{awscdkdocs} & 2.142.1 & Framework utilizado para definição dos componentes de infraestrutura como código \\

\hline

SST \citep{sstdocs} & 2.43.0 & Framework para construção de aplicações full-stack na AWS utilizando infraestrutura como código \\

\hline

Docusaurus \citep{docusaurusdocs} & \textsuperscript{$\wedge$}3.4.0 & Gerador de documentação em formato de site estático. \\

\hline

React \citep{reactdocs} & \textsuperscript{$\wedge$}18.0.0 & Biblioteca utilizada no desenvolvimento do cliente web do sistema \\

\hline

Stoplight Elements \citep{stoplightdocs} & \textsuperscript{$\wedge$}8.3.1 & Biblioteca utilizada para renderização da documentação da API. \\

\hline

Docker \citep{dockerdocs} & 24.0.5 & Ferramenta utilizada para a criação do ambiente isolado do gateway de conexão \\

\hline

Chakra UI \citep{chakrauidocs} & \textsuperscript{$\wedge$}2.8.2 & Biblioteca de componentes de interface utilizada no cliente web. \\

\hline

Apache Guacamole \citep{apacheguacamoledocs} & 1.5.3 & Utilizado para intermediar a conexão entre o cliente web e gateway de conexão. \\

\hline

Vite \citep{vitedocs} & \textsuperscript{$\wedge$}5.0.12 & Ferramenta responsável por gerenciar a implementação e o empacotamento do cliente web \\

\hline

MongoDB Atlas \citep{mongodbatlasdocs} & sdk \textsuperscript{$\wedge$}6.0.3 & Serviço de banco de dados não relacional utilizado para armazenar todos os dados do sistema. \\

\hline

AWS Amplify \citep{awsamplifydocs} & sdk \textsuperscript{$\wedge$}6.0.16 & Biblioteca com integrações facilitadas, uitilizada no cliente web para autenticação e conexão com AWS AppSync \\

\hline

AWS Lambda PowerTools \citep{awslambdapowertools} & \textsuperscript{$\wedge$}2.1.1 & Biblioteca utilizada para integrar a geração de logs dos serviços com o AWS CloudWatch \\

\hline

AWS AppSync \citep{awsappsync} & sdk \textsuperscript{$\wedge$}3.592.0 & Serviço utilizado para a comunicação em tempo real entre os serviçoes de backend e o cliente web \\

\hline

AWS Systems Manager \citep{awssystemsmanagerdocs} & sdk \textsuperscript{$\wedge$}3.592.0 & Serviço utilizado para armazenar as configurações do sistema. \\

\hline

AWS Event Bridge \citep{awseventbridgedocs} & sdk \textsuperscript{$\wedge$}3.592.0 & Serviço de gerenciamento de eventos utilizado como barramento de comunicação entre os serviços de backend. Oferece também a funcionalidade de agendamento de eventos, utilizado no sistema de desligamento automático de instâncias osciosas. \\

\hline

AWS CloudFormation \citep{awscloudformationdocs} & sdk \textsuperscript{$\wedge$}3.592.0 & Serviço utilizado para armazenar a definição dos componentes de infraestrutura como código \\

\hline

AWS Cognito \citep{awscognito} & sdk \textsuperscript{$\wedge$}3.592.0 & Serviço de gerenciamento de usuários e credenciais. Ele foi utilizado para gerenciar dados sensíveis, mantendo somente dados operacionais na integração com o banco de dados. \\

\hline

AWS Service Catalog \citep{awsservicecatalogdocs} & sdk \textsuperscript{$\wedge$}3.592.0 & Serviço de gerenciamento de modelos de IaC utilizado para o controle do ciclo de vida da instância e seus recursos dependentes \\

\hline

AWS CloudWatch \citep{awscloudwatchdocs} & sdk \textsuperscript{$\wedge$}3.592.0 & Serviço de monitoramento utilizado para agregar os logs dos sistemas. \\

\hline

AWS Lambda \citep{awslambdadocs} & sdk \textsuperscript{$\wedge$}3.592.0 & Serviço de computação serverless utilizado para executar o serviço de API e tratamento de eventos do sistema \\

\hline

AWS Api Gateway \citep{awsapigatewaydocs} & sdk \textsuperscript{$\wedge$}3.592.0 & Serviço gerenciado para criação de APIs serverless. \\

\hline

AWS Identity and Access Management \citep{awsiamdocs} & sdk \textsuperscript{$\wedge$}3.592.0 & Serivço utilizado para delimitação de todas as permissões atreladas aos componentes de infraestrutura do sistema \\

\hline

AWS S3 \citep{awss3docs} & sdk \textsuperscript{$\wedge$}3.592.0 & Serviço de armazenamento de objetos utilizado para servir o cliente web e armazenar todos os arquivos imutaveis do sistema \\

\hline

AWS CloudFront \citep{awscloudfrontdocs} & sdk \textsuperscript{$\wedge$}3.592.0 & Serviço de distribuição de conteúdo utilizado como camada distribuída de cache para os serviços \\

\hline

AWS Certificate Manager \citep{awscertificatemanagerdocs} & sdk \textsuperscript{$\wedge$}3.592.0 & Serviços utilizado para o gerenciamento dos certificados de domínio atrelados ao cliente web e à documentação \\

\hline

AWS Elastic Load Balancing \citep{awselasticloadbalancing} & sdk \textsuperscript{$\wedge$}3.592.0 & Balanceador de carga gerenciado utilizado para gerenciar as conexões entre o cliente web e o cluster de containers rodando o Gateway de conexão. \\

\hline

AWS Elastic Container Service \citep{awsecsdocs} & sdk \textsuperscript{$\wedge$}3.592.0 & Serviço utilizado para a implantação do gateway de conexão, no formato de cluster de containeres com escalabilidade previsível \\

\hline

AWS Elastic Container Registry \citep{awsecrdocs} & sdk \textsuperscript{$\wedge$}3.592.0 & Serviço utilizado para armazenar as imagens de container do gateway de conexão \\

\hline

AWS Simple Notification Service \citep{awssnsdocs} & sdk \textsuperscript{$\wedge$}3.592.0 & Serviço de publisher Subscriber utilizado para a transacionar as mensagens emitidas na criação de instâncias \\

\hline

AWS EC2 \citep{awsec2docs} & sdk \textsuperscript{$\wedge$}3.592.0 & Serviço central utilizado para gerenciar as instâncias, regras de conexão e as imagens de sistema operacional \\

\hline

\end{longtable}

O ecossistema de tecnologias utilizado foi escolhido a partir de alguns critérios, como a familiaridade com o ambiente de desenvolvimento, a facilidade de integração entre as ferramentas e a disponibilidade de documentação e comunidade de suporte.

Visto que a \gls{utfpr} já possui integração com a \gls{aws}, a escolha da provedor de nuvem pública foi trivial, já que a infraestrutura de serviços da \gls{aws} é também utilizada em projetos a nível empresarial e acadêmico a nível global, oferecendo um portfólio extenso de serviços que atendem as necessidades do sistema.

O SST foi escolhido como um facilitador tanto para o fluxo de desenvolvimento local quanto para a definição de todos os componentes de arquiteura como código. A ferramenta disponibiliza componentes de alto nível que já implementam as melhores práticas de segurança e controle de custos. Além disso, o SST estabelece uma conexão entre os serviços da \gls{aws}, como as funções Lambda, e o código escrito localmente, fazendo com que seja possível testar e depurar o código no seu ambiente próprio de implantação.

O \gls{guacamole} foi escolhido como intermediador da conexão entre o cliente web e as instâncias de máquinas virtuais, porque trata-se de uma ferramenta de código aberto bem estabelecida que fornece os componentes necessários para a conexão remota a partir de um navegador de internet. Além disso, ele é compatível com os protocolos de conexão remota utilizados no trabalho. No contexto do desenvolvimento do sistema, o \gls{guacamole} foi utilizado de forma customizada. Apenas \textit{daemon} que implementa a conexão remota, o componente web que renderiza a interface de conexão e o protocolo de comunicação em si foram utilizados.

Essa customização foi necessária para que o sistema pudesse controlar todos os aspectos de interface e conexão, caso contrário, seria necessário utilizar a aplicação web já existente do \gls{guacamole}. Se esse fosse o caso, a customização do sistema dependeria de um conjunto de regras de negócio implementadas diretamente no \gls{guacamole}, o que dificultaria a manutenção e a evolução do sistema.


\section{Implementa\c{c}\~ao}
\label{sec:implementacao}



TODO

\section{Testes e Valida\c{c}\~ao}
\label{sec:testesEValidacao}

Os testes do sistema foram realizados de duas formas, a primeira através de testes unitários e a segunda através da execução manual de um teste \textit{end-to-end} percorrendo os caminhos críticops da aplicação. (TODO, Refereciar)

Os testes unitários foram implementados através do framework \textit{Jest} a fim validar de forma satisfatória todos os casos de uso dos serviços de backend. Como visto anteriormente, a estrutura do projeto já foi projetada para facilitar a escrita de testes utilizando \glspl{adapter} implementados especificamente para reproduzir o comportamento dos serviços reais da camada de infraestrutura.

A seguir os testes unitários dos casos de usos mais importantes implementados serão listados, com suas condições iniciais e resultados esperados. O \autoref{cap:anexoa} apresenta o relatório de cobertura dos testes com os resultados obtidos.

\begin{table}[h]
\caption{Testes unitários do caso de uso: Desligar instância ociosa}
\label{tab:testeDesligarInstanciaOciosa}
\begin{tabularx}{\textwidth}{p{0.45\textwidth} p{0.5\textwidth}}
\toprule
\textbf{Condição inicial} & \textbf{Resultado esperado} \\ \midrule

Quando a entrada do caso de uso é inválida & Deve retornar um erro de validação \\ \hline

Quando a instância indicada não existe & Não deve realizar nenhuma ação \\ \hline

Quando a instância indicada existe e pertence a um usuário & Deve desligar a instância \\ 

\bottomrule
\end{tabularx}
\fonte{Autoria própria (2024)}%% Fonte
\end{table}

Os casos de teste da \autoref{tab:testeDesligarInstanciaOciosa} são importantes para garantir que quando o agendamento para desligamento é acionado, a instância de fato será desligada, evitando assim que recursos sejam desperdiçados e que o usuário consiga acessar a instância de outro local diferente do cliente web do sistema.


\begin{table}[h]
\caption{Testes unitários do caso de uso: Deletar instância}
\label{tab:testeDeletarInstancia}
\begin{tabularx}{\textwidth}{p{0.45\textwidth} p{0.5\textwidth}}
\toprule
\textbf{Condição inicial} & \textbf{Resultado esperado} \\ \midrule

Quando a entrada do caso de uso é inválida & Deve retornar um erro de validação \\ \hline

Quando o usuário não tem permissão para deletar a instância & Deve retornar um erro de permissão \\ \hline

Quando a instância indicada não existe & Não retornar um erro de instância não encontrada \\ \hline

Quando o cargo do usuário não é administrador e a instância pertence a outro usuário & Deve retornar um erro de permissão \\ \hline

Quando o usuário tem permissão para deletar a instância & Deve deletar a instância e todos os recursos associados \\

\bottomrule
\end{tabularx}
\fonte{Autoria própria (2024)}%% Fonte
\end{table}

Na \autoref{tab:testeDeletarInstancia} os testes também são importantes para o gerenciamento correto dos recursos da nuvem, já que é preciso deletar o registro do objeto no banco de dados e também todos os recursos que foram criados na nuvem, como a instância, o grupo de segurança, volumes de armazenamento, estes que se ficarem foram de uso, podem gerar custos desnecessários e difíceis de rastrear.


\begin{table}[h]
\caption{Testes unitários do caso de uso: Obter informações de conexão com instância} 
\label{tab:testeObterInformacoesDeConexaoComInstancia}
\begin{tabularx}{\textwidth}{p{0.45\textwidth} p{0.5\textwidth}}
\toprule
\textbf{Condição inicial} & \textbf{Resultado esperado} \\ \midrule

Quando a entrada do caso de uso é inválida & Deve retornar um erro de validação \\ \hline

Quando o usuário não está habilitado para acessar o sistema & Deve retornar um erro de permissão \\ \hline

Quando a instância indicada não existe & Deve retornar um erro de instância não encontrada \\ \hline

Quando o sistema não consegue obter a senha da instância & Deve retornar um erro interno \\ \hline

Quando o usuário não é o dono da instância e não é administrador & Deve retornar um erro de permissão \\ \hline

Quando o usuário tem acesso à instância e a mesma ainda não está pronta para conexão & Deve retornar um erro de violação do caso de uso \\ \hline

Quando o usuário tem acesso à instância e a mesma não está ligada & Deve retornar um erro de violação do caso de uso \\ \hline

Quando o usuário tem acesso à instâcia e o protocolo interno de conexão é \gls{rdp} & Deve retornar as informações de conexão com a instância \\ \hline

Quando o usuário tem acesso à instâcia e o protocolo interno de conexão é \gls{vnc} & Deve retornar as informações de conexão com a instância \\

\bottomrule
\end{tabularx}
\fonte{Autoria própria (2024)}%% Fonte
\end{table}

Os testes da \autoref{tab:testeObterInformacoesDeConexaoComInstancia} garantem dentre outras coisas,que só será possível iniciar uma conexão com a instância se a mesma estiver de fato preparada para receber conexões. Para chegar nesse estado, a instância precisa passar por várias etapas que sem a validação correta, ficariam praticamente impossíveis de serem rastreadas e corrigidas.


\begin{table}[h]
\caption{Testes unitários do caso de uso: Criar instância}
\label{tab:testeCriarInstancia}
\begin{tabularx}{\textwidth}{p{0.45\textwidth} p{0.5\textwidth}}
\toprule
\textbf{Condição inicial} & \textbf{Resultado esperado} \\ \midrule

Quando a entrada do caso de uso é inválida & Deve retornar um erro de validação \\ \hline

Quando o usuário não está habilitado para acessar o sistema & Deve retornar um erro de permissão \\ \hline

Quando o usuário não é encontrado no sistema & Deve retornar um erro de usuário não encontrado \\ \hline

Quando o template de instância indicado não existe & Deve retornar um erro de template não encontrado \\ \hline

Quando o tipo de hardware indicado não existe & Deve retornar um erro de tipo de hardware não encontrado \\ \hline

Quando a imagem de sistema operacional indicada não existe & Deve retornar um erro de imagem não encontrada \\ \hline

Quando o cargo do usuário não é administrador e o dono indicado é diferente do usuário & Deve retornar um erro de permissão \\ \hline

Quando o usuário não é administrador e o limite de instâncias do usuário foi atingido & Deve retornar um erro de violação do caso de uso \\ \hline

Quando o usuário não é administrador e o tipo de hardware indicado não é permitido para o usuário & Deve retornar um erro de violação do caso de uso \\ \hline

Quando o usuário não é administrador e a opção de hibernação não é permitida para o usuário & Deve retornar um erro de violação do caso de uso \\ \hline

Quando o usuário tem permissão para criar a instância & Deve criar a instância e retornar as informações da instância \\

\bottomrule
\end{tabularx}
\fonte{Autoria própria (2024)}%% Fonte
\end{table}

Os testes da \autoref{tab:testeCriarInstancia} são importantes para garantir que a criação de instâncias seja feita de forma segura e controlada, evitando que usuários mal intencionados possam criar instâncias sem permissão ou que usuários comuns possam criar instâncias com configurações que não são permitidas para o seu cargo.

\begin{table}[h]
\caption{Testes unitários do caso de uso: Notificar mudança de estado de instância}
\label{tab:testeNotificarMudancaEstadoInstancia}
\begin{tabularx}{\textwidth}{p{0.45\textwidth} p{0.5\textwidth}}
\toprule
\textbf{Condição inicial} & \textbf{Resultado esperado} \\ \midrule

Quando a entrada do caso de uso é inválida & Deve retornar um erro de validação \\ \hline

Quando a instância indicada não existe & Não deve realizar nenhuma ação \\ \hline

Quando o usuário dono da instância não existe & Não deve realizar nenhuma ação \\ \hline

Quando a instância recebe o estado de ligada & Deve sinalizar o sistema de desligamento automático e notificar o usuário dono da instância sobre a mudança de estado \\ \hline

Quando a instância recebe qualquer outro estado & Deve notificar o usuário dono da instância sobre a mudança de estado \\

\bottomrule
\end{tabularx}
\fonte{Autoria própria (2024)}%% Fonte
\end{table}

Os testes da \autoref{tab:testeNotificarMudancaEstadoInstancia} são importantes para garantir que o sistema notifique os usuários sobre as mudanças de estado das instâncias, permitindo a atualização da interface do cliente web em tempo real evitando que o usuário tenha que recarregar a página para visualizar as mudanças.

\begin{table}[h]
\caption{Testes unitários do caso de uso: Criar template de instância a partir de instância}
\label{tab:testesCriarTemplateDeInstanciaAPartirDeInstancia}
\begin{tabularx}{\textwidth}{p{0.45\textwidth} p{0.5\textwidth}}
\toprule
\textbf{Condição inicial} & \textbf{Resultado esperado} \\ \midrule

Quando a entrada do caso de uso é inválida & Deve retornar um erro de validação \\ \hline

Quando o usuário não é administrador & Deve retornar um erro de permissão \\ \hline

Quando a instância indicada não existe & Deve retornar um erro de instância não encontrada \\ \hline

Quando a instância utilizada como base ainda não terminou de ser configurada & Deve retornar um erro de violação do caso de uso \\ \hline

Quando a quantidade de armazenamento indicada é menor que o armazenamento mínimo requerido pela imagem de sistema operacional & Deve retornar um erro de violação do caso de uso \\ \hline

Quando todas as condições são atendidas & Deve criar o template de instância e retornar as informações do template \\

\bottomrule
\end{tabularx}
\fonte{Autoria própria (2024)}%% Fonte
\end{table}

Os testes da \autoref{tab:testesCriarTemplateDeInstanciaAPartirDeInstancia} são importantes para garantir que os administradores possam criar templates de instâncias a partir de instâncias já criadas. As verificações nesse caso de uso garantem que os usuários terão templates válidos para criar instâncias que funcionem corretamente.

\begin{table}[h]
\caption{Testes unitários do caso de uso: Cadastrar usuário}
\label{tab:testecadastrarusuario}
\begin{tabularx}{\textwidth}{p{0.45\textwidth} p{0.5\textwidth}}
\toprule
\textbf{Condição inicial} & \textbf{Resultado esperado} \\ \midrule

Quando a entrada do caso de uso é inválida & Deve retornar um erro de validação \\ \hline

Quando já existe um usuário no banco de dados com o \textit{username} recebido na entrada & Deve retornar o usuário já cadastrado \\ \hline

Quando não existir um registro no banco de dados com o \textit{username} recebido na entrada & Deve criar o usuário e retornar as informações do usuário \\

Quando o tipo de hardware de instância padrão indicado não existe & Deve retornar um erro interno \\

\bottomrule
\end{tabularx}
\fonte{Autoria própria (2024)}%% Fonte
\end{table}

Os testes da \autoref{tab:testecadastrarusuario} validam se a sincronia entre o \textit{AWS Congito} e o banco de dados do sistema prevê casos de borda para que não haja duplicidade de usuários no sistema e os mesmo possam ser criados de forma controlada, tanto através do cliente web, quanto através do \textit{console} ou \textit{cli} da AWS.
    



\section{Implanta\c{c}\~ao do sistema}
\label{sec:implantacaoDoSistema}

