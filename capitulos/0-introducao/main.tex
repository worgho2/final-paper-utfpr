%%%% CAPÍTULO 1 - INTRODUÇÃO
%%
%% Deve apresentar uma visão global da pesquisa, incluindo: breve histórico, importância e justificativa da escolha do tema,
%% delimitações do assunto, formulação de hipóteses e objetivos da pesquisa e estrutura do trabalho.

%% Título e rótulo de capítulo (rótulos não devem conter caracteres especiais, acentuados ou cedilha)
\chapter{Introdução}\label{cap:introducao}

Em 1997 o professor de sistemas de informação Ramnath Challappa utilizou o termo Computação em nuvem pela primeira vez para representar um conceito tecnológico que já era utilizado desde os anos 1950. Antes de ganhar o nome mais atual, a tecnologia chegou a ser chamada de \textit{Utility Computing}, fazendo alusão ao acesso compartilhado e de forma simultânea a computadores mainframe, que ocupavam muitas vezes salas completas e custavam caro. \citep{dellcloud}

A computação em nuvem atualmente já aparece como pilar de soluções tecnológicas em situações de escalas variadas. Por um lado, uma grande empresa pode utilizá-la para prover a infraestrutura de grandes aplicações de operação crítica, e por outro lado, um desenvolvedor pode executar cargas de trabalhos em uma instância remota, desativá-la após concluir e pagar apenas pelo tempo de uso do recurso. \citep{taurioncloud}

\textit{VDI} ou Infraestrutura de \textit{desktops} Virtuais é o termo usado para referenciar um ambiente de infraestrutura que hospeda \textit{desktops} virtuais e provê para usuários finais a medida que são requisitados. Tanto ambientes \textit{on premise} quanto em nuvem utilizam a mesma base conceitual de virtualização para tal arquitetura, a grande diferença está na forma em que os recursos são precificados. Em servidores locais, o recurso deve ser comprado previamente, gerenciado e protegido, já em ambientes de nuvem, onde o modelo de responsabilidade compartilhada exime o usuário de responsabilidades de segurança e gerenciamento físicos dos recursos o preço é pago pela utilização dos recursos. \citep{vmwarevdi}

Embora tenham sido feitos progressos utilizando \textit{VDI} em nuvem e a influência dos fatores de qualidade de serviço em sistemas de ensino baseados em \textit{VDI} \citep{qoselearning}, poucos estudos foram realizados com o fim de apresentar uma infraestrutura completa de aplicação que sustente \textit{desktops} virtuais em nuvem pública para utilização em ensino e pesquisa.

O objetivo deste trabalho é propor um formato de infraestrutura de \textit{desktops} virtuais em nuvem pública para sustentar ensino e pesquisa, materializado através de um protótipo de aplicação que garante o acesso à recursos computacionais a partir de qualquer dispositivo com um navegador de internet instalado.

Através da aplicação, os responsáveis pela infraestrutura disponibilizarão templates de imagens\footnote{Templates de images são cópias do sistema operacional com todas as configurações já estabelecidas. Eles facilitam a configuração de um novo \textit{desktop}, reduzindo o tempo até estarem acessíveis para o usuário final} que podem ser usadas. Os professores poderão autorizar e gerenciar quais tipos de imagens, os alunos podem acessar. Por fim os alunos poderão acessar os recursos computacionais de qualquer lugar, dentro ou fora da universidade.

No cenário de uma aula de laboratório onde um conjunto de softwares é necessário para a execução das atividades e os mesmos demandam uma capacidade computacional maior do que os dispositivos físicos dos alunos é capaz de fornecer, ainda sim todos seriam capazes de participar, já que esses dispositivos servem como interface para os recursos computacionais em nuvem.

Em outro cenário, também será possível a um pesquisador acessar recursos computacionais com dados persistidos entre sessões. 

\section{Objetivos}\label{sec:objetivos}

Na seção a seguir, os objetivos que norteiam o desenvolvimento do trabalho serão apresentados na forma de um objetivo principal e objetivos específicos que delimitam as etapas do projeto.

\subsection{Objetivo geral}\label{subsec:objetivoGeral}

O objetivo do trabalho é propor um formato de infraestrutura de \textit{desktops} virtuais em nuvem pública para sustentar ensino e pesquisa.

\subsection{Objetivos específicos}\label{subsec:objetivosEspecificos}

\begin{itemize}
    \item Analisar o desempenho dos protocolos de conexão com \textit{desktops} remotos para definir quais protocolos serão utilizados.

    \item Elaborar um diagrama de arquitetura de aplicação em nuvem.

    \item Desenvolver um protótipo de aplicação baseado no diagrama de arquitetura, com segmentação de acesso por tipo de usuário, onde alunos acessam \textit{desktops} virtuais através de navegadores de internet, professores gerenciam os recursos que os alunos têm acesso e os administradores de sistema disponibilizam templates de máquina que podem ser utilizados dentro da aplicação.

    \item Levantar uma estimativa de custo do protótipo de aplicação construído, levando em consideração aspectos de utilização mensal dos recursos de núvem.
\end{itemize}

\section{Justificativa}\label{sec:justificativa}

Em 2021, o censo de educação superior mostrou que cerca de 3,9 milhões de estudantes ingressaram em instituições de ensino superior e que 77,7\% destes concluíram o ensino médio na rede pública de ensino. \citep{inep2021}
Para aqueles que não possuem computadores portáteis de uso próprio, ainda é um desafio lidar com demandas que exigem o acesso a esse tipo de equipamento.

No contexto da UTFPR, os alunos têm a possibilidade de utilizar os equipamentos em aulas de laboratório ou na biblioteca. Porém a pandemia de COVID-19 evidenciou que essas políticas de acesso ainda não são suficientes para promover o acesso igualitário entre os estudantes.

Esse trabalho apresenta uma proposta de interesse compartilhado. Universidades públicas que dependem de licitação para adquirir equipamentos, otimizarão o uso de seus recursos, uma vez que a solução do trabalho requisita dispositivos com configurações modestas para acesso, abrindo caminho para economia de custos com equipamentos e também para reutilização de computadores com capacidade reduzida, além de promover o acesso facilitado à recursos em nuvem dimensionáveis para pesquisa. E alunos terão a possibilidade de acesso a recursos computacionais através de qualquer dispositivo com acesso à internet e que tenha um navegador web.


\section{Estrutura do trabalho}\label{sec:estruturaTrabalho}

O trabalho está estruturado em três capítulos. O Capítulo 1 apresenta o assunto do trabalho, bem como o tema e os objetivos da produção. O Capítulo 2 detalha os conceitos utilizados para a concepção do projeto, o estado da arte e mostra outros trabalhos e pesquisas que já foram publicados e que estão relacionados com o tema do atual trabalho. Por fim, o Capítulo 3 descreve quais materiais e métodos serão empregados para o cumprimento dos objetivos propostos, bem como o cronograma de atividades a serem realizadas.