%%%% CAPÍTULO 1 - INTRODUÇÃO
%%
%% Deve apresentar uma visão global da pesquisa, incluindo: breve histórico, importância e justificativa da escolha do tema,
%% delimitações do assunto, formulação de hipóteses e objetivos da pesquisa e estrutura do trabalho.

%% Título e rótulo de capítulo (rótulos não devem conter caracteres especiais, acentuados ou cedilha)
\chapter{Introdução}
\label{cap:introducao}
% OK

As origens do termo Computação em nuvem datam do início dos anos 1990, quando componentes desconhecidos em diagramas de redes de computadores eram representados por nuvens. Mesmo não significando exatamente o que é hoje, a ideia de que a nuvem representava algo que estava fora das fronteiras de uma rede local já estava presente. Antes de ganhar o nome mais atual, a tecnologia chegou a ser chamada de \textit{Utility Computing}, fazendo alusão ao acesso compartilhado e de forma simultânea a computadores \textit{mainframe}, que ocupavam muitas vezes salas completas e custavam caro. O termo só chegou formalmente ao mercado em 2006, quando a Amazon lançou o serviço de computação em nuvem \textit{Elastic Compute Cloud} (EC2). \citep{cloudcomputingcambridge}

A computação em nuvem atualmente já aparece como pilar de soluções tecnológicas em situações de
escalas variadas. Por um lado, uma grande empresa pode utilizá-la para prover a infraestrutura de
grandes aplicações de operação crítica, e por outro lado, um desenvolvedor pode executar cargas de
trabalhos em uma instância remota, desativá-la após concluir e pagar apenas pelo tempo de uso do
recurso. \citep{taurioncloud}

Infraestrutura de Desktops Virtuais, do inglês \textit{Virtual Desktop Infrastructure} é o termo usado para referenciar um ambiente de infraestrutura que hospeda \glspl{desktop} virtuais e provê para usuários finais a medida que são requisitados. Tanto ambientes \gls{onPremise}, onde os servidores estão localizados na infraestrutura local, quanto em nuvem utilizam a mesma base conceitual de virtualização para tal arquitetura, a grande diferença está na forma em que os recursos são precificados. Em servidores locais, o recurso deve ser adquirido previamente, gerenciado e protegido, já em ambientes de nuvem, onde o modelo de responsabilidade compartilhada exime o usuário de responsabilidades de segurança e gerenciamento físicos dos recursos, o preço é pago pela utilização dos recursos. \citep{cloudcomputingcambridge}

Embora tenham sido feitos progressos na utilização de \gls{vdi} em nuvem e no estudo da influência dos fatores de qualidade de serviço em sistemas de ensino baseados em \gls{vdi} \citep{qoselearning}, poucos estudos foram realizados a fim de apresentar uma infraestrutura completa de aplicação que sustente \glspl{desktop} virtuais em nuvem pública para utilização em ensino e pesquisa no contexto de instituições públicas de ensino superior.

O objetivo deste trabalho é o desenvolvimento de um sistema de \gls{vdi} em nuvem pública que permita o acesso a \glspl{desktop} virtuais a partir de qualquer dispositivo com um navegador de internet instalado, promovendo o acesso a recursos computacionais adequadamente robustos para a execução de atividades acadêmicas e de pesquisa, a partir de qualquer lugar, dentro ou fora da universidade. Além disso, viabilizar o consumo dos mesmos recursos computacionais como serviço, reduzindo custos com aquisição e manutenção de equipamentos e garantindo a disponibilidade de recursos computacionais para alunos e pesquisadores.

O acesso ao sistema baseado em papéis, permitirá que diferentes combinações de sistemas operacionais, \textit{softwares} e especificações de \textit{hardware} sejam disponibilizados para públicos com necessidades específicas, como alunos de diferentes cursos e pesquisadores de diferentes áreas de atuação.

Através de uma interface \textit{web} principal, os professores poderão criar configurações de \glspl{desktop} virtuais contendo todos os requisitos necessários para a execução de atividades acadêmicas e de pesquisa e disponibilizá-los para os alunos. Estes, por sua vez, poderão utilizar as configurações disponibilizadas para criar suas instâncias e escolher entre diferentes tipos de \textit{hardware} previamente habilitados pelos professores.

O sistema também lidará com a persistência de dados entre sessões, desligamento preemptivo de recursos ociosos e também com todo o ciclo de vida dos recursos de infraestrutura, garantindo que os recursos sejam desalocados após o uso e que novos recursos sejam alocados conforme a demanda, de forma rápida e escalável.


% No cenário de uma aula de laboratório onde um conjunto de softwares é necessário para a execução das
% atividades e os mesmos demandam uma capacidade computacional maior do que os dispositivos físicos
% dos alunos é capaz de fornecer, ainda sim todos seriam capazes de participar, já que esses
% dispositivos servem como interface para os recursos computacionais em nuvem.

% Em outro cenário, também será possível a um pesquisador acessar recursos computacionais com dados
% persistidos entre sessões. 

\section{Objetivos}
\label{sec:objetivos}
% OK

A seguir, o objetivo geral e os objetivos específicos do trabalho são apresentados. O objetivo geral
é o propósito principal do trabalho, que através dos objetivos específicos é desdobrado em partes
menores que enriquecem o escopo do trabalho.

\subsection{Objetivo geral}
\label{subsec:objetivoGeral}
% OK

Desenvolver um sistema de \gls{vdi} em nuvem pública capaz de prover acesso a \glspl{desktop}
virtuais para ensino e pesquisa a partir de qualquer dispositivo com um navegador de internet
instalado.

\subsection{Objetivos específicos}
\label{subsec:objetivosEspecificos}
% OK

\begin{itemize}
    \item Modelar a infraestrutura do sistema de \gls{vdi}, utilizando serviços de computação em nuvem pública, garantindo escalabilidade, alta disponibilidade e segurança dos dados.

    \item Modelar o sistema utilizando práticas de desenvolvimento de \textit{software} que garantam a extensibilidade do sistema validada através da cobertura de testes unitários dos casos de uso.

    \item Implementar a \gls{iac} para todos os componentes do sistema, permitindo a implantação completa em novos ambientes de forma automatizada.

    \item Implementar um sistema de configuração para que sejam possível conectar um provedor de identidade e autenticação ao sistema no momento da implantação.

    \item Apresentar uma estimativa de custo de utilização do sistema contemplando diferentes cenários de utilização.

    \item Desenvolver uma documentação externa, que aborde tanto a configuração do sistema em novos ambientes quanto o uso do sistema através da interface \textit{web} principal ou da \gls{api} que o sistema disponibiliza.
\end{itemize}

\section{Justificativa}
\label{sec:justificativa}
% OK

Promover o acesso a recursos computacionais de forma remota, a partir de qualquer dispositivo com acesso à internet e que tenha um navegador \textit{web}, de forma que os alunos e pesquisadores possam acessar recursos computacionais de qualquer lugar, através de dispositivos com capacidade computacional limitada.

\section{Motivação}

Em 2021, o censo de educação superior mostrou que cerca de 3,9 milhões de estudantes ingressaram em instituições de ensino superior e que 77,7\% destes concluíram o ensino médio na rede pública de ensino. \citep{inep2021}
Para aqueles que não possuem computadores de uso próprio, ainda é um desafio lidar com demandas acadêmicas que exigem o acesso a esse tipo de equipamento dentro e fora da universidade.

No contexto da \gls{utfpr}, os alunos têm a possibilidade de utilizar os equipamentos em aulas de laboratório ou na biblioteca. Porém a pandemia de COVID-19 evidenciou que essas alternativas ainda não são suficientes para promover o acesso igualitário entre os estudantes. Este cenário é ainda mais restrito para pesquisadores que necessitam de recursos computacionais mais robustos para executar suas atividades.

Não menos importante, o trabalho também visa possibilidade de redução de custos com aquisição de equipamentos para universidades públicas, como também elucida uma abordagem de consumo de recursos computacionais como serviço, onde os recursos robustos são utilizados pelo tempo necessário, sem a necessidade de aquisição, manutenção e tratativas relacionadas à depreciação de equipamentos.

\section{Estrutura do trabalho}
\label{sec:estruturaTrabalho}
% OK

Este trabalho está organizado em cinco capítulos, sendo este o primeiro deles, a introdução. Seguido do \autoref{cap:revisaoDaLiteratura} que apresenta a revisão da literatura, o \autoref{cap:metodologia} que apresenta a metodologia, o \autoref{cap:resultados} que apresenta os resultados obtidos e por fim o \autoref{cap:conclusao} que apresenta as conclusões obtidas com o desenvolvimento do trabalho e discute sobre perspectivas futuras para o trabalho.

