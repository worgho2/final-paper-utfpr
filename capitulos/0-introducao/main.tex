%%%% CAPÍTULO 1 - INTRODUÇÃO
%%
%% Deve apresentar uma visão global da pesquisa, incluindo: breve histórico, importância e justificativa da escolha do tema,
%% delimitações do assunto, formulação de hipóteses e objetivos da pesquisa e estrutura do trabalho.

%% Título e rótulo de capítulo (rótulos não devem conter caracteres especiais, acentuados ou cedilha)
\chapter{Introdução}\label{cap:introducao}

(REVISAR INTRODUÇÃO)

Em 1997 o professor de sistemas de informação Ramnath Challappa utilizou o termo Computação em nuvem
pela primeira vez para representar um conceito tecnológico que já era utilizado desde os anos 1950.
Antes de ganhar o nome mais atual, a tecnologia chegou a ser chamada de \textit{Utility Computing},
fazendo alusão ao acesso compartilhado e de forma simultânea a computadores mainframe, que ocupavam
muitas vezes salas completas e custavam caro. \citep{dellcloud}

A computação em nuvem atualmente já aparece como pilar de soluções tecnológicas em situações de
escalas variadas. Por um lado, uma grande empresa pode utilizá-la para prover a infraestrutura de
grandes aplicações de operação crítica, e por outro lado, um desenvolvedor pode executar cargas de
trabalhos em uma instância remota, desativá-la após concluir e pagar apenas pelo tempo de uso do
recurso. \citep{taurioncloud}

\gls{vdi} é o termo usado para referenciar um ambiente de infraestrutura que hospeda \glspl{desktop}
virtuais e provê para usuários finais a medida que são requisitados. Tanto ambientes \gls{onPremise}
quanto em nuvem utilizam a mesma base conceitual de virtualização para tal arquitetura, a grande
diferença está na forma em que os recursos são precificados. Em servidores locais, o recurso deve
ser comprado previamente, gerenciado e protegido, já em ambientes de nuvem, onde o modelo de
responsabilidade compartilhada exime o usuário de responsabilidades de segurança e gerenciamento
físicos dos recursos o preço é pago pela utilização dos recursos. \citep{vmwarevdi}

Embora tenham sido feitos progressos utilizando \gls{vdi} em nuvem e a influência dos fatores de
qualidade de serviço em sistemas de ensino baseados em \gls{vdi} \citep{qoselearning}, poucos
estudos foram realizados com o fim de apresentar uma infraestrutura completa de aplicação que
sustente \glspl{desktop} virtuais em nuvem pública para utilização em ensino e pesquisa.

O objetivo deste trabalho é propor um formato de infraestrutura de \glspl{desktop} virtuais em
nuvem pública para sustentar ensino e pesquisa, materializado através de um protótipo de aplicação
que garante o acesso à recursos computacionais a partir de qualquer dispositivo com um navegador de
internet instalado.

Através da aplicação, os responsáveis pela infraestrutura disponibilizarão templates de
imagens\footnote{Templates de images são cópias do sistema operacional com todas as configurações já
estabelecidas. Eles facilitam a configuração de um novo \gls{desktop}, reduzindo o tempo até
estarem acessíveis para o usuário final} que podem ser usadas. Os professores poderão autorizar e
gerenciar quais tipos de imagens, os alunos podem acessar. Por fim os alunos poderão acessar os
recursos computacionais de qualquer lugar, dentro ou fora da universidade.

No cenário de uma aula de laboratório onde um conjunto de softwares é necessário para a execução das
atividades e os mesmos demandam uma capacidade computacional maior do que os dispositivos físicos
dos alunos é capaz de fornecer, ainda sim todos seriam capazes de participar, já que esses
dispositivos servem como interface para os recursos computacionais em nuvem.

Em outro cenário, também será possível a um pesquisador acessar recursos computacionais com dados
persistidos entre sessões. 

\section{Objetivos}
\label{sec:objetivos}
% OK

A seguir, o objetivo geral e os objetivos específicos do trabalho são apresentados. O objetivo geral
é o propósito principal do trabalho, que através dos objetivos específicos é desdobrado em partes
menores que enriquecem o escopo do trabalho.

\subsection{Objetivo geral}
\label{subsec:objetivoGeral}
% OK

Desenvolver um sistema de \gls{vdi} em nuvem pública capaz de prover acesso a \glspl{desktop}
virtuais para ensino e pesquisa a partir de qualquer dispositivo com um navegador de internet
instalado.

\subsection{Objetivos específicos}
\label{subsec:objetivosEspecificos}
% OK

\begin{itemize}
    \item Modelar a infraestrutura do sistema de \gls{vdi}, utilizando serviços de computação em nuvem pública, garantindo escalabilidade, alta disponibilidade e segurança dos dados.

    \item Modelar o sistema utilizando práticas de desenvolvimento de software que garantam a extensibilidade do sistema validada através da cobertura de testes unitários dos casos de uso.

    \item Implementar a \gls{iac} para todos os componentes do sistema, permitindo a implantação completa em novos ambientes de forma automatizada.

    \item Implementar um sistema de configuração para que sejam possível conectar um provedor de identidade e autenticação ao sistema no momento da implantação.

    \item Apresentar uma estimativa de custo de utilização do sistema contemplando diferentes cenários de utilização.

    \item Desenvolver uma documentação externa, que aborde tanto a configuração do sistema em novos ambientes quanto o uso do sistema através da interface \textit{web} principal ou da \gls{api} que o sistema disponibiliza.
\end{itemize}

\section{Justificativa}
\label{sec:justificativa}
% OK

Em 2021, o censo de educação superior mostrou que cerca de 3,9 milhões de estudantes ingressaram em
instituições de ensino superior e que 77,7\% destes concluíram o ensino médio na rede pública de
ensino. \citep{inep2021}
Para aqueles que não possuem computadores de uso próprio, ainda é um desafio lidar com demandas
acadêmicas que exigem o acesso a esse tipo de equipamento dentro e fora da universidade.

No contexto da \gls{utfpr}, os alunos têm a possibilidade de utilizar os equipamentos em aulas de
laboratório ou na biblioteca. Porém a pandemia de COVID-19 evidenciou que essas alternativas ainda
não são suficientes para promover o acesso igualitário entre os estudantes. Este cenário é ainda mais
restrito para pesquisadores que necessitam de recursos computacionais mais robustos para executar
suas atividades.

Portanto, a principal motivação para a realização deste trabalho é a possibilidade de promover o
acesso a recursos computacionais de forma remota, a partir de qualquer dispositivo com acesso à
internet e que tenha um navegador web, de forma que os alunos e pesquisadores possam acessar
recursos computacionais de qualquer lugar, através de dispositivos com capacidade computacional
limitada.

Não menos importante, o trabalho também visa a redução de custos com aquisição de equipamentos para
universidades públicas e uma abordagem de consumo de recursos computacionais como serviço, que
não restringe o acesso a recursos robustos a processos licitatórios e aquisições de equipamentos.


\section{Estrutura do trabalho}
\label{sec:estruturaTrabalho}
% OK

Este trabalho está organizado em cinco capítulos, conforme a seguir:

\begin{itemize}
    \item \textbf{\autoref{cap:introducao} - Introdução}: apresenta o assunto do trabalho, bem como o tema e os objetivos da produção.

    \item \textbf{\autoref{cap:revisaoDaLiteratura} - Revisão da Literatura}: detalha os conceitos utilizados para a concepção do projeto, o estado da arte e mostra outros trabalhos e pesquisas que já foram publicados e que estão relacionados com o tema do atual trabalho.

    \item \textbf{\autoref{cap:metodologia} - Metodologia}: descreve quais materiais e métodos serão empregados para o cumprimento dos objetivos propostos.

    \item \textbf{\autoref{cap:resultados} - Resultados}: apresenta os resultados obtidos com a execução do projeto e discute os aspectos relevantes.

    \item \textbf{\autoref{cap:conclusao} - Conclusão}: apresenta as conclusões obtidas com o desenvolvimento do trabalho e sugere aprimeiramentos e trabalhos futuros.
\end{itemize}
