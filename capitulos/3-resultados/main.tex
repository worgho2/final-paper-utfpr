%%%% CAPÍTULO 4 - RESULTADOS E DISCUSSÃO

\chapter{Resultados}
\label{cap:resultados}

Este capítulo apresenta aspectos da implementação do sistema proposto, bem como as discussões acerca
das métricas obtidas durante os testes realizados.

O capítulo está divido em sete seções.
A \autoref{sec:escopoSistema} busca considerar aspectos técnicos e conceituais do sistema afim de delimitar o escopo do sistema.
Na \autoref{sec:modelagemSistema} os diagramas e processos de modelagem do sistema são apresentados
seguindo as definições formais apresentadas no \autoref{cap:materiaisemetodos}.
A \autoref{sec:apresentacaoSistema} apresenta o sistema do ponto de vista do usuário, focando nas funcionalidades e interações com o sistema,
já a \autoref{sec:implementacaoSistema} foca em nos aspectos técnicos mais relevantes na implementação do sistema.
A \autoref{sec:implantacaoSistema} apresenta os cenários de implantação do sistema, trazendo perspectivas de uso e capacidade de escala.
Na \autoref{sec:discussoes} aspectos relevantes dos resultados de testes e discussões são apresentados.
Por fim, os trabalhos e produções tecnológicas que buscam resolver problemas semelhantes ao proposto neste trabalho são apresentados na \autoref{sec:trabalhosRelacionados}.

% Este capítulo apresenta o que foi obtido como resultado do trabalho, que, em princípio, é o sistema desenvolvido. Se não for um sistema, como, por exemplo, uma solução na área de redes, neste capítulo é reportada a solução proposta. Neste caso, a divisão do capítulo em seções é realizada, se necessária, de acordo com o trabalho.

% O capítulo pode conter seções de acordo com o tipo de sistema e a necessidade de documentação mais extensa de determinados aspectos. Caso o trabalho se refira à comparação entre tecnologias ou dados obtidos como resultados do uso do sistema, além da descrição do sistema, há os dados obtidos com os testes e a discussão desses dados. Nesse caso haverá uma seção para os dados obtidos desses testes e as discussões.

\section{Escopo do sistema}
\label{sec:escopoSistema}

O sistema \textit{Virtual Lab} deve possibilidar a criação e gerenciamento de instâncias de máquinas virtuais para uso em laboratórios remotos de ensino e pesquisa.
O acesso deve ser feito por meio de um navegador de internet, permitindo a utilização de qualquer dispositivo com um navagador de internet com acesso à rede.

O sistema deve acomodar dois tipos de usuários: usuários comuns e administradores.
Os usuários comuns devem ter permissão de criar instâncias de máquinas virtuais com base em templates pré-definidos, gerenciar suas próprias instâncias, bem como o conteúdo instalado nas mesmas.
Os administradores devem ter as mesmas permissões dos usuários comuns, além de gerenciar os templates de instância, controlar o acesso dos usuários comuns ao sistema e aos recursos de hardware.

O sistema deve permitir o acesso aos usuários através de autenticação por meio de usuário e senha, bem como através de um provedor de identidade externo, caso o ambiente de uso do sistema já possua um provedor de identidade.

Inicialmente, um usuário administrador deve criar os templates de instância indicando o sistema operacional e a quantidade de armazenamento disponível.
Os usuários comuns devem então criar instâncias com base nesses templates, escolhendo entre os tipos de hardware permitidos.

Ao ser criada, a instância deve passar por um processo de configuração inicial para garantir que o sistema operacional esteja pronto para uso, e então ser disponibilizada para o usuário, que poderá acessá-la atráves da página de conexões do sistema. As instâncias devem ser encerradas automaticamente após um período de inatividade.

\section{Modelagem do sistema}
\label{sec:modelagemSistema}

TODO

- diagrama 

% A modelagem do sistema inclui os diagramas e as descrições textuais para representar o problema e a solução.

% Sendo assim, primeiramente esse item deve apresentar diagramas utilizados para a modelagem de negócios (ex. diagramas de atividade e estado), se esses tenham sido necessários.
% Em seguida esse item deve conter a descrição dos requisitos obtidos do usuário, contendo sua respectiva classificação (funcionais e não funcionais). Sugere-se o uso de um modelo formal sugerido por autores (ex. Wazlawick, Bezerra) para a apresentação dessa classificação.

% Se utilizada orientação a objetos e a UML, nesta seção ainda são apresentados, por exemplo, os diagramas de casos de uso, com suas descrições suplementares, os diagramas de classe de análise (ou modelo conceitual), de sequência e/ou comunicação, diagrama de classes de projeto.

% Nesta seção também estão os diagramas da modelagem de banco de dados, como entidade-relacionamento. Nesse item pode ser apresentada a descrição de cada uma das classes do modelo de classes apresentado acima, assim como a descrição das tabelas do banco de dados. Também podem estar documentados modelos e padronizações utilizados para a interface, diagramas de navegação, a representação da arquitetura do sistema e dos padrões de projeto utilizados.

\section{Apresentação do sistema}
\label{sec:apresentacaoSistema}

TODO

% Apresenta as funcionalidades e o uso de recursos tecnológicos do sistema por meio de suas telas, enfatizando a interação com o sistema. A apresentação do sistema é feita sob a forma de texto, com telas e definição de padrões que forem relevantes ao contexto do trabalho. As telas são tratadas como figuras, cópias (print screen) de relatórios ou consultas também são figuras.

% A \autoref{fig:cadastroPaciente} exibe a tela de acesso ao Cadastro de Pacientes.

% \begin{figure}[htpb]%% Ambiente figure
% \captionsetup{width=0.43\textwidth}
% \caption{Tela de acesso ao Cadastro de Pacientes.}%% Legenda
% \label{fig:cadastroPaciente}%% Rótulo
% \includegraphics[scale=0.8]{cadastro-paciente}%% Dimensões e localização
% \fonte{}%% Fonte
% \end{figure}

\section{Implementação do sistema}
\label{sec:implementacaoSistema}

TODO

% Nesta seção é documentada a implementação do sistema com partes relevantes ou exemplos de código, rotinas, funções. Inclui, ainda, a descrição técnica do uso de recursos (componentes, bibliotecas, etc.) da linguagem. Ressalta-se que cada orientador avaliará juntamente com seu orientado o que poderá ser descrito nesta seção. Isso sem que sejam revelados detalhes do sistema que possam comprometer seu uso comercial ou científico ou que a descrição fique muito sucinta ou superficial.

% Em materiais e método estão quais os recursos utilizados, neste capítulo é reportado como esses recursos foram utilizados para resolver o problema.

% Sugere-se colocar listagens curtas de código, enfatizando aspectos específicos das tecnologias utilizadas ou da implementação. Sugere-se, ainda, que o código não seja apresentado sob a forma de print screen, e sim copiado e colado no texto, mantendo, se possível, a formatação. Todas as listagens de código devem ser devidamente explicadas. A explicação deve ser técnica, fundamentada em aspectos conceituais e boas práticas de programação.

% Enfatizar os diferenciais do sistema: procedimentos armazenados, consultas SQL, uso de componentes, uso de padrões de projeto, a forma de uso dos recursos da linguagem. Esses diferenciais são no sentido de explicitar as vantagens, desvantagens, dificuldades e facilidades que esses recursos impetraram no desenvolvimento do sistema em termos técnicos. Esses diferenciais servirão para avaliar pela utilização ou não desses recursos, pelo menos para sistemas iguais ou semelhantes ao reportado no trabalho.

% Reportar a forma como o sistema foi verificado e validado. No sentido de verificar se os requisitos definidos para o mesmo foram atendidos. Os testes podem ser realizados pelo professor orientador, pelos professores que compõem a banca, por pessoas que serviram de base para as informações para o sistema e etc. Os testes podem ser realizados com base em um plano de testes elaborado juntamente com a análise e projeto do sistema. Para validar a implementação podem ser desenvolvidas rotinas de teste unitário.

% Se houver implantação do sistema, mesmo que seja para teste, reportar a forma como isso foi feito, a geração de instaladores, os problemas com ambiente e sistema operacional, incluindo banco de dados e outros. Deixar explícito o procedimento para instalar e usar o sistema.

% Quando for necessário, citar no texto do trabalho nomes de campos, tabelas ou rotinas específicas utilizadas na implementação de um software, utilizar a fonte courier new para destacar esses nomes.

% Um exemplo de listagem de código fonte pode ser observado na \autoref{codigo:classeFoo}, que representa a classe Aluno.

% \begin{sourcecode}[htb]
% \caption{\label{codigo:classeFoo}Classe Aluno}
% \begin{lstlisting}[frame=single, language=Java]
% @Entity
% public class Foo {
 
%     @Id
%     @GeneratedValue(strategy = GenerationType.IDENTITY)
%     private Long id;
 
%     private String nome;
    
%     private Integer ra;
     
%     // constructor, getters and setters
% }
% \end{lstlisting}
% \fonte{}
% \end{sourcecode}

\section{Implantação do sistema}\label{sec:implantacaoSistema}

TODO

\section{Discussões}\label{sec:discussoes} 

TODO

% O trabalho contém esta seção quando considerado que há resultados (em termos de dados) e discussões relevantes ou suficientes para justificar uma seção. Se existentes e não justificarem uma seção, eles podem estar na seção que relata a implementação do sistema.

% Nesta seção estão os resultados obtidos da realização de testes quantitativos e qualitativos, independentemente da quantidade, tipo e volume de testes realizados. Os resultados dos testes são discutidos tendo como base o referencial teórico e os objetivos pretendidos com o trabalho. Esses testes podem resultar de implantação e testes de uso do sistema. 

\section{Trabalhos Relacionados}
\label{sec:trabalhosRelacionados}

O trabalho desenvolvido por \citet{qoselearning} realiza uma análise quantitativa do impacto de
parâmetros de conexão de rede em ambientes de ensino remoto sustentados por uma estrutura de \gls{vdi}.
A análise é feita a partir de um experimento com o objetivo de clarificar a relação entre a qualidade de
conexão com a internet e a usabilidade de sistemas de ensino remoto.

O resultado do experimento mostrou que é possível realizar atividades como escrita, desenhos e
consumo de mídias, sem grandes prejuízos para a usabilidade, mas é necessário que a conexão com a
internet tenha qualidade. Outro ponto importante é que fora atividades de visualização de vídeos, a
largura de banda não é um fator de grande consumo nesses tipos de ambientes.

Outro trabalho de grande relevância para essa produção, desenvolvido por \citet{edufirestick},
apresenta um sistema de \gls{vdi} para educação remota utilizando as mesmas tecnologias propostas
no presente trabalho. 

A proposta da produção é de criar uma \gls{vdi} de baixo custo com \glspl{ec2SpotInstance} da
\gls{aws} e com o \gls{guacamole} para gerenciamento das conexões, e permitindo o acesso de
qualquer dispositivo com um navegador de internet, até mesmo através televisões com acesso à internet,
como apresentado no trabalho.

A solução proposta por \citet{edufirestick} aborda um cenário de aula de laboratório, onde um
professor pode agendar um evento de criação das máquinas virtuais para os alunos e no horário da
aula, todos poderiam acessar o recurso reservado para a aula.

Em termos de custos operacionais, o trabalho apresenta uma estimativa de custo por usuário de
USD\$ 0.87 mensais com a utilização dos seguintes parâmetros:

\begin{itemize}
    \item Tempo de conexão diária: 18 horas
    \item Categoria da instância: t3.micro
    \item Sistema operacional: \textit{Microsoft Windows Server 2019}
    \item Memória: 4 GB de RAM
    \item Espaço em disco: 90 GB
    \item Região: São Paulo
\end{itemize}

Em termos de preço pela transferência de dados, o trabalho relata um custo de USD\$ 0.06 por hora em
picos onde o uso de banda chaga a 2 MB/s. O custo dos outros serviços que suportam a infraestrutura,
foi estimado em USD\$ 200.00 mensais.

O trabalho feito por \citet{edufirestick} apresenta um panorama muito promissor e consoante aos
objetivos do presente trabalho, demostrando que é possível executar a solução em um cenário de
laboratório. Mas ainda não apresenta resultados referentes ao gerenciamento de \glspl{desktop} que
tenham os dados persistidos, para outros cenários de uso, como o de pesquisa por exemplo.