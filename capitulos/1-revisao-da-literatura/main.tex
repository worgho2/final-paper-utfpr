%%%% CAPÍTULO 2 - REVISÃO DA LITERATURA (OU REVISÃO BIBLIOGRÁFICA, ESTADO DA ARTE, ESTADO DO CONHECIMENTO)
%%
%% O autor deve registrar seu conhecimento sobre a literatura básica do assunto, discutindo e comentando a informação já publicada.
%% A revisão deve ser apresentada, preferencialmente, em ordem cronológica e por blocos de assunto, procurando mostrar a evolução do tema.
%% Título e rótulo de capítulo (rótulos não devem conter caracteres especiais, acentuados ou cedilha)
\chapter{revis\~ao Da Literatura}
\label{cap:revisaoDaLiteratura}
% OK

Este capítulo apresenta a base teórica fundamental para o desenvolvimento deste trabalho.

% Computação em nuvem
\section{Computa\c{c}\~ao em Nuvem}
\label{sec:computacaoEmNuvem}

TODO referenciar livro de cloud computing

% IaaS
\section{Infraestrutura como Servi\c{c}o}
\label{sec:infraestruturaComoServico}

TODO referenciar livro de cloud computing

% Arquitetura hexagonal
\section{Arquitetura Hexagonal}
\label{sec:arquiteturaHexagonal}

TODO REFERENCIAR (RICHARDS)


\section{Trabalhos Relacionados}
\label{sec:trabalhosRelacionados}

O trabalho desenvolvido por \citet{qoselearning} realiza uma análise quantitativa do impacto de
parâmetros de conexão de rede em ambientes de ensino remoto sustentados por uma estrutura de \gls{vdi}.
A análise é feita a partir de um experimento com o objetivo de clarificar a relação entre a qualidade de
conexão com a internet e a usabilidade de sistemas de ensino remoto.

O resultado do experimento mostrou que é possível realizar atividades como escrita, desenhos e
consumo de mídias, sem grandes prejuízos para a usabilidade, mas é necessário que a conexão com a
internet tenha qualidade. Outro ponto importante é que fora atividades de visualização de vídeos, a
largura de banda não é um fator de grande consumo nesses tipos de ambientes.

Outro trabalho de grande relevância para essa produção, desenvolvido por \citet{edufirestick},
apresenta um sistema de \gls{vdi} para educação remota utilizando as mesmas tecnologias propostas
no presente trabalho. 

A proposta da produção é de criar uma \gls{vdi} de baixo custo com \glspl{ec2SpotInstance} da
\gls{aws} e com o \gls{guacamole} para gerenciamento das conexões, e permitindo o acesso de
qualquer dispositivo com um navegador de internet, até mesmo através televisões com acesso à internet,
como apresentado no trabalho.

A solução proposta por \citet{edufirestick} aborda um cenário de aula de laboratório, onde um
professor pode agendar um evento de criação das máquinas virtuais para os alunos e no horário da
aula, todos poderiam acessar o recurso reservado para a aula.

Em termos de custos operacionais, o trabalho apresenta uma estimativa de custo por usuário de
USD\$ 0.87 mensais com a utilização dos seguintes parâmetros:

\begin{itemize}
    \item Tempo de conexão diária: 18 horas
    \item Categoria da instância: t3.micro
    \item Sistema operacional: \textit{Microsoft Windows Server 2019}
    \item Memória: 4 GB de \gls{ram}
    \item Espaço em disco: 90 GB
    \item Região: São Paulo
\end{itemize}

Em termos de preço pela transferência de dados, o trabalho relata um custo de USD\$ 0.06 por hora em
picos onde o uso de banda chaga a 2 MB/s. O custo dos outros serviços que suportam a infraestrutura,
foi estimado em USD\$ 200.00 mensais.

O trabalho feito por \citet{edufirestick} apresenta um panorama muito promissor e consoante aos
objetivos do presente trabalho, demostrando que é possível executar a solução em um cenário de
laboratório. Mas ainda não apresenta resultados referentes ao gerenciamento de \glspl{desktop} que
tenham os dados persistidos, para outros cenários de uso, como o de pesquisa por exemplo.