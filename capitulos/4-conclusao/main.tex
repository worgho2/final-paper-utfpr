%%%% CAPÍTULO 5 - CONCLUSÕES E PERSPECTIVAS
%%
\chapter{Conclus\~ao}\label{cap:conclusao}

TODO
% Inicia com um resumo do trabalho, retomando o(s) objetivo(s), o referencial teórico e o uso das ferramentas e das tecnologias utilizadas no trabalho.

% A conclusão contém a opinião do autor em relação às vantagens, desvantagens, facilidades e limitações das tecnologias e/ou do método utilizados, as dificuldades encontradas e como foram superadas.

% Também devem ser apresentadas as vantagens, desvantagens e limitações do trabalho desenvolvido, sempre tendo em vista a sua contribuição para a comunidade acadêmica e profissional e para a sociedade como um todo.

% É a opinião técnica do autor do trabalho em relação ao assunto sob a forma de uma espécie de avaliação em relação ao trabalho desenvolvido e as tecnologias utilizadas.

% Finaliza verificando se o objetivo foi alcançado e com a opinião do autor sobre o assunto, de acordo com o referencial teórico e com os resultados obtidos.


\section{Perspectivas Futuras}
\label{sec:perspectivasFuturas}
% OK

A partir do trabalho desenvolvido, é possível identificar pontos que podem ser aprimorados ou mesmo novas funcionalidades que podem ser implementadas. A seguir, são apresentadas algumas sugestões de trabalhos futuros:

\begin{itemize}
    \item Implantação de um \textit{proxy} de rede entre as instâncias gerenciadas pelo sistema e a internet, de forma a aplicar filtros de segurança e monitoramento de tráfego. Assim como é feito no ambiente de rede da \gls{utfpr}.

    \item Adição da funcionalidade de compartilhamento de sessão entre usuários, já que o protocolo base de comunicação do \gls{guacamole} suporta essa funcionalidade. 

    \item Implantação de um serviço de compartilhamento de arquivos em rede entre as instâncias do sistema, com acesso isolado para cada usuário.

    \item Implementação de um mecanismo para possibilitar o redimensionamento dos recursos de \textit{hardware} das instâncias gerenciadas pelo sistema, como quantidade de armazenamento, memória e processamento.
\end{itemize}